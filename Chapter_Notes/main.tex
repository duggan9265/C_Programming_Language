\documentclass{article}

\usepackage[a4paper, margin=1in]{geometry}

\usepackage[T1]{fontenc}
\usepackage[utf8]{inputenc}

\usepackage{listings}
\usepackage{xcolor}

\definecolor{codegray}{rgb}{0.5,0.5,0.5}
\definecolor{codepurple}{rgb}{0.58,0,0.82}
\definecolor{backcolour}{rgb}{0.95,0.95,0.92}

\lstdefinestyle{CStyle}{
  backgroundcolor=\color{backcolour},   
  commentstyle=\color{codegray}\itshape,
  keywordstyle=\color{blue}\bfseries,
  numberstyle=\tiny\color{codegray},
  stringstyle=\color{codepurple},
  basicstyle=\ttfamily\footnotesize,
  breakatwhitespace=false,         
  breaklines=true,                 
  captionpos=b,                    
  keepspaces=true,                 
  numbers=left,                    
  numbersep=5pt,                  
  showspaces=false,                
  showstringspaces=false,
  showtabs=false,                  
  tabsize=2,
  language=C
}

\lstset{style=CStyle}
 


\title{C Programming Language: Chapter Notes}
\author{Daniel Duggan}
\date{\today}

\begin{document}
\maketitle

% ------------------ Chapter 2: Input and Output Functions ----------------------------------

\section{Chapter 2 Notes: Input and Output Functions}

\noindent C provides powerful yet simple functions for performing input and output operations. Two of the most commonly used are \texttt{printf()} and \texttt{scanf()}, both defined in the \texttt{<stdio.h>} header file.

\subsection{The \texttt{printf()} Function}

\noindent The \texttt{printf()} function is used to send formatted output to the standard output stream (usually the screen). Its general form is:

\begin{lstlisting}[caption={General form of printf()}, label={code:printf}]
  printf("control string", expressions);
\end{lstlisting}

\noindent The control string contains ordinary characters, which are printed as they appear, and conversion specifications, each beginning with a \texttt{\%} character, which tell \texttt{printf()} how to format and display the corresponding expressions. 

\begin{center}
\begin{tabular}{ |c|c| } 
 \hline
 \textbf{Specifier} & \textbf{Meaning}  \\ 
 \%d & prints an integer value  \\ 
 \%f & prints a floating-point value  \\ 
 \%c & prints a single character \\ 
 \%s & prints a string of characters \\ 
 \%02d & prints an integer padded with zeros (e.g. 05) \\ 
 \%.2f & prints a floating-point number to 2 decimal places \\ 
 \hline
\end{tabular}
\end{center}

\noindent For example:

\begin{lstlisting}[caption={Example of printf() usage}, label={code:printf-example}]
#include <stdio.h>

int main(void) {
    int hour = 9, minute = 5;
    float temperature = 18.437;

    printf("The time is %d:%02d\n", hour, minute);
    printf("Temperature: %.2f\\textdegree C\n", temperature);

    return 0;
}
\end{lstlisting}

\noindent This will print:
\begin{verbatim}
The time is 9:05
Temperature: 18.44°C
\end{verbatim}

\noindent Here, \texttt{\%02d} tells \texttt{printf()} to print the integer using two digits, padding with a leading zero if necessary, while \texttt{\%.2f} rounds the floating-point number to two decimal places.

\subsection{The \texttt{scanf()} Function}

\noindent The \texttt{scanf()} function reads formatted input from the standard input stream (usually the keyboard). Its general form is:

\begin{lstlisting}[caption={General form of scanf()}, label={code:scanf}]
  scanf("control string", address list);
\end{lstlisting}

\noindent The control string works similarly to that of \texttt{printf()}, using format specifiers to determine the type of data being read. Each variable to be read must be preceded by an \texttt{\&} (address-of operator), because \texttt{scanf()} needs to know where to store the input value.

\begin{center}
\begin{tabular}{ |c|c| } 
 \hline
 \textbf{Specifier} & \textbf{Meaning}  \\ 
 \%d & reads an integer value  \\ 
 \%f & reads a floating-point value  \\ 
 \%c & reads a single character \\ 
 \%s & reads a string of characters (stops at whitespace) \\ 
 \hline
\end{tabular}
\end{center}

\noindent For example:

\begin{lstlisting}[caption={Example of scanf() usage}, label={code:scanf-example}]
#include <stdio.h>

int main(void) {
    int hour, minute;

    printf("Enter a 24-hour time (hh:mm): ");
    scanf("%d:%d", &hour, &minute);

    printf("You entered %02d:%02d\n", hour, minute);

    return 0;
}
\end{lstlisting}

\noindent If the user inputs \texttt{09:30}, the program will output:
\begin{verbatim}
You entered 09:30
\end{verbatim}

\noindent The \texttt{scanf()} format string \texttt{"\%d:\%d"} expects the colon literally in the input, so the user must type it (e.g. \texttt{09:30}). The colon can appear directly in the control string when you want the input to match a specific pattern.

\subsection{Important Notes and Common Pitfalls}

\begin{itemize}
  \item \textbf{Always match format specifiers to variable types.}  
  Using the wrong one (for example, reading a float with \texttt{\%d}) can lead to undefined behaviour.

  \item \textbf{Use \texttt{\&} in \texttt{scanf()}.}  
  Forgetting it causes the program to overwrite memory at an undefined location, often leading to crashes.  
  \texttt{scanf("\%d", n);} is incorrect — it should be \texttt{scanf("\%d", \&n);}.

  \item \textbf{Whitespace handling:}  
  \texttt{scanf()} skips leading whitespace (spaces, newlines, tabs) for most format specifiers except \texttt{\%c}, which reads the next character literally — even if it’s a newline.

  \item \textbf{\texttt{\%s} stops at whitespace.}  
  It reads characters until the first space, tab, or newline. To read full lines of text including spaces, use \texttt{fgets()} instead.

  \item \textbf{Input validation:}  
  If the user types a non-numeric value when \texttt{\%d} or \texttt{\%f} is expected, \texttt{scanf()} fails and leaves variables unchanged. Always check its return value (the number of items successfully read).

  \item \textbf{Clearing the input buffer:}  
  After using \texttt{scanf()}, unwanted newline characters can remain in the input buffer and affect the next input.  
  To safely handle mixed input (e.g. reading both numbers and characters), use:
  \begin{lstlisting}
  while (getchar() != '\n'); // clear leftover characters
  \end{lstlisting}

  \item \textbf{Printing special characters:}  
  To print a percent sign, use \texttt{\%\%}. For example:
  \begin{lstlisting}
  printf("Progress: 100%% complete\n");
  \end{lstlisting}
\end{itemize}

\subsection{Summary}

\begin{itemize}
  \item \texttt{printf()} is used for formatted output; \texttt{scanf()} for formatted input.
  \item Both functions use format specifiers beginning with \texttt{\%} to handle different data types.
  \item Each variable in \texttt{scanf()} must be preceded by \texttt{\&}.
  \item \texttt{scanf()} skips whitespace for most specifiers but not for \texttt{\%c}.
  \item \texttt{printf()} allows control of output precision and padding with \texttt{0} and field width.
  \item Always ensure the types of variables match their format specifiers.
\end{itemize}

% ------------------ Chapter 5: Selection Statements ----------------------------------

\section{Chapter 5 Notes: Selection Statements}

\subsection{Relational Operators}

\begin{center}
\begin{tabular}{ |c|c| } 
 \hline
 \textbf{Symbol} & \textbf{Meaning}  \\ 
 $<$ & less than  \\ 
 $>$ & greater than  \\
 $<=$ & less than or equal to \\
 $>=$ & greater than or equal to\\ 
 \hline
\end{tabular}
\end{center}

\noindent The relational operators produce 0 when false and 1 when true e.g. \texttt{1 $<$ 2} produces 1 (true). \texttt{1$>$ 2} produces 0 (false).

\noindent The expression \texttt{i $<$ j $<$ k} doesn´t have the meaning one may expect, as $<$ is left associative. It is equivalent to \texttt{(i$<$j)$<$k} i.e. it tests whether i is less than j, and then the 1 or 0 is compared with the value of k.

\noindent To test if j lies within i and k, the correct expression is \texttt{i$<$j \&\& j$<$k}.

\subsection{Equality Operators}

\begin{center}
\begin{tabular}{ |c|c| } 
 \hline
 \textbf{Symbol} & \textbf{Meaning}  \\ 
 == & equal to  \\ 
 != & not equal to  \\ 
 \hline
\end{tabular}
\end{center}

\noindent Like the relational operators, the equality operators are left associative and produce 0 or 1.  However, they have \textit{lower precedence} than the relational operators, meaning the expression 
\texttt{i$<$j == j$<$k} is equivalent to \texttt{(i$<$j) == (j$<$k)}.

\subsection{Logical Operators}

\begin{center}
\begin{tabular}{ |c|c| } 
 \hline
 \textbf{Symbol} & \textbf{Meaning}  \\ 
 ! & logical negation  \\ 
 \&\& & logical \textit{and}  \\ 
 $||$ & logical \textit{or}\\
 \hline
\end{tabular}
\end{center}

\noindent The logical operators often produce 0 or 1 as there result. They behave as follows:

\begin{itemize}
  \item \texttt{!expr} has the value 1 if \textit{expr} has the value 0.
  \item \texttt{expr1 \&\& expr2} has the value 1 if both expressions are non-zero.
  \item \texttt{expr1 || expr2} has the value 1 if either \texttt{expr1} or \texttt{expr2} (or both) has a non-zero value.
\end{itemize}

\subsection{if statement}

An if statement takes the form

\begin{lstlisting}[caption={if statement}, label={code:if statement}]
  if (expression) statement
\end{lstlisting}

\noindent If you want numeroues statements, one can use the curly braces:

\begin{lstlisting}
  if (line_num == max_lines){

    line_num = 0;
    page_num++;
  }
\end{lstlisting}

\noindent Curly braces can be add to if and if else statements even when not strictly required to aid in readability:

\begin{lstlisting}[caption={Using curly braces for readability in if-else statements}, label={lst:ifelsebraces}]
#include <stdio.h>

int main(void) {
    int score = 85;

    // Even though braces aren't strictly required for single statements,
    // adding them improves clarity and helps prevent future errors.
    if (score >= 90) {
        printf("Grade: A\n");
    } else if (score >= 80) {
        printf("Grade: B\n");
    } else {
        printf("Grade: C or below\n");
    }

    return 0;
}
\end{lstlisting}

\subsubsection{Dangling else Problem}

\noindent An else statement belongs to the nearest if statement that hasn´t already been paired with an else. Using curly braces aids in recognising which else statement belongs to which if statement.

\subsection{Conditional Expressions}

\noindent C provides an operator that allows an expression to produce one of two values depending on the value of a condition. This operator cosists of two symbols (? and :) which have to be used in the following way.

\begin{lstlisting}[caption={Using curly braces for readability in if-else statements}, label={lst:ifelsebraces}]
  expr1 ? expr2 : expr3
\end{lstlisting}

\noindent The expressions can be of any type. This operator is unique in C in that it requires 3 operands instead of the usual one or two and is therefore called the ternary operator. It is read as \textit{"if expr1 thhen expr2 else expr3"}.

\noindent expr1 is evaluated, if it is not 0, then expr2 is evaluated and it´s value is the value of the expression. If expr1 is 0, then the value of expr3 is the value of the expression as a whole. For example:

\begin{lstlisting}
  i = 1;
  j = 2;
  k = i>j ? i : j; // if i is greater than j, k is value of i (1), else it is value of  j (2).
                  // so here, k is 2
  k = (i>=0 ? i : 0) + j; // k is now 3 (i is greater than 0, so expr2 is evaluated, and has +j added to it)
\end{lstlisting}

\subsection{Boolean Types}

\noindent None in C89. Need to define a macro ifelsebraces
\begin{lstlisting}
  #define BOOL int
  .
  .
  .
  BOOL flag
\end{lstlisting}

\noindent In C99 there are boolean types declared basicstyle
\begin{lstlisting}
  _Bool flag
\end{lstlisting}

\noindent where $\_Bool$ is an unsigned integer type that can only be assigned 0 or 1. Giving value greater than 1 usually causes it to be assigned 1. The header file $<stdbool.h>$ can also be used which makes it easier to work with boolean values. It provides a macro, bool, that stands for $\_Bool$.

\subsection{Switch Statement}

\noindent Similar to a case statement in VHDL/Ada. Its syntax is:

\begin{lstlisting}
  switch (expression) {
    case constant-expression : statements
    ... 
    case constant-exprsion : statements
    default : statements
  }
\end{lstlisting}

\noindent For example:


\begin{lstlisting}

  int grade = 2;
  
switch (grade) {
    case 1:
        printf("Grade: A\n");
        break;
    case 2:
        printf("Grade: B\n");
        break;
    case 3:
        printf("Grade: C\n");
        break;
    default:
        printf("Grade: F (fail)\n");
        break;
}

\end{lstlisting}

\noindent Here, the value of the variable grade is tested against different values. As int grade = 2, when case 2 is executed, the statement inside this case will execute. Each case ends with a break; to prevent "fall-through". Without the break;, the next case is evaluated even if the condition doesn´t match.


\end{document}