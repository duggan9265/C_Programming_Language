\documentclass{article}
\usepackage[a4paper, margin=1in]{geometry}

\title{C Programming Language: Chapter Notes}
\author{Daniel Duggan}
\date{\today}

\begin{document}
\maketitle

\section{Chapter 5 Notes}

\subsection{Relational Operators}

\begin{center}
\begin{tabular}{ |c|c| } 
 \hline
 \textbf{Symbol} & \textbf{Meaning}  \\ 
 $<$ & less than  \\ 
 $>$ & greater than  \\
 $<=$ & less than or equal to \\
 $>=$ & greater than or equal to\\ 
 \hline
\end{tabular}
\end{center}

\noindent The relational operators produce 0 when false and 1 when true e.g. \texttt{1 $<$ 2} produces 1 (true). \texttt{1$>$ 2} produces 0 (false).

\noindent The expression \texttt{i $<$ j $<$ k} doesn´t have the meaning one may expect, as $<$ is left associative. It is equivalent to \texttt{(i$<$j)$<$k} i.e. it tests whether i is less than j, and then the 1 or 0 is compared with the value of k.

\noindent To test if j lies within i and k, the correct expression is \texttt{i$<$j \&\& j$<$k}.

\subsection{Equality Operators}

\begin{center}
\begin{tabular}{ |c|c| } 
 \hline
 \textbf{Symbol} & \textbf{Meaning}  \\ 
 == & equal to  \\ 
 != & not equal to  \\ 
 \hline
\end{tabular}
\end{center}

\noindent Like the relational operators, the equality operators are left associative and produce 0 or 1.  However, they have \textit{lower precedence} than the relational operators, meaning the expression 
\texttt{i$<$j == j$<$k} is equivalent to \texttt{(i$<$j) == (j$<$k)}.

\subsection{Logical Operators}

\begin{center}
\begin{tabular}{ |c|c| } 
 \hline
 \textbf{Symbol} & \textbf{Meaning}  \\ 
 ! & logical negation  \\ 
 \&\& & logical \textit{and}  \\ 
 $||$ & logical \textit{or}\\
 \hline
\end{tabular}
\end{center}

\noindent The logical operators often produce 0 or 1 as there result. They behave as follows:

\begin{itemize}
  \item \texttt{!expr} has the value 1 if \textit{expr} has the value 0.
  \item \texttt{expr1 \&\& expr2} has the value 1 if both expressions are non-zero.
  \item \texttt{expr1 || expr2} has the value 1 if either \texttt{expr1} or \texttt{expr2} (or both) has a non-zero value.
\end{itemize}


\end{document}